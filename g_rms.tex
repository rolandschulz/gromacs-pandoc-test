\documentclass[]{article}
\usepackage[T1]{fontenc}
\usepackage{lmodern}
\usepackage{amssymb,amsmath}
\usepackage{ifxetex,ifluatex}
\usepackage{fixltx2e} % provides \textsubscript
% use microtype if available
\IfFileExists{microtype.sty}{\usepackage{microtype}}{}
\ifnum 0\ifxetex 1\fi\ifluatex 1\fi=0 % if pdftex
  \usepackage[utf8]{inputenc}
\else % if luatex or xelatex
  \usepackage{fontspec}
  \ifxetex
    \usepackage{xltxtra,xunicode}
  \fi
  \defaultfontfeatures{Mapping=tex-text,Scale=MatchLowercase}
  \newcommand{\euro}{€}
\fi
\usepackage{ctable}
\usepackage{float} % provides the H option for float placement
\ifxetex
  \usepackage[setpagesize=false, % page size defined by xetex
              unicode=false, % unicode breaks when used with xetex
              xetex]{hyperref}
\else
  \usepackage[unicode=true]{hyperref}
\fi
\hypersetup{breaklinks=true,
            bookmarks=true,
            pdfauthor={},
            pdftitle={},
            colorlinks=true,
            urlcolor=blue,
            linkcolor=magenta,
            pdfborder={0 0 0}}
\setlength{\parindent}{0pt}
\setlength{\parskip}{6pt plus 2pt minus 1pt}
\setlength{\emergencystretch}{3em}  % prevent overfull lines
\setcounter{secnumdepth}{0}

\author{}
\date{}

\begin{document}

\subsection{g\_rms}

\textbf{VERSION 4.5.5-dev-20110920-5282f-unknown\\ Sun 12 Aug 2012}

\begin{center}\rule{3in}{0.4pt}\end{center}

\subsubsection{Description}

\textbf{g\_rms} compares two structures by computing the root mean
square deviation (RMSD), the size-independent $\rho$ similarity
parameter (\textbf{rho}) or the scaled $\rho$ (\textbf{rhosc}), see
Maiorov \& Crippen, Proteins \textbf{22}, 273 (1995). This is selected
by \textbf{-what}.

Each structure from a trajectory (\textbf{-f}) is compared to a
reference structure. The reference structure is taken from the structure
file (\textbf{-s}).

With option \textbf{-mir} also a comparison with the mirror image of the
reference structure is calculated. This is useful as a reference for
`significant' values, see Maiorov \& Crippen, Proteins \textbf{22}, 273
(1995).

Option \textbf{-prev} produces the comparison with a previous frame the
specified number of frames ago.

Option \textbf{-m} produces a matrix in \hyperref[xpm]{xpm} format of
comparison values of each structure in the trajectory with respect to
each other structure. This file can be visualized with for instance
\textbf{xv} and can be converted to postscript with
\hyperref[xpm2ps]{xpm2ps}.

Option \textbf{-fit} controls the least-squares fitting of the
structures on \hyperref[top]{top} of each other: complete fit (rotation
and translation), translation only, or no fitting at all.

Option \textbf{-mw} controls whether mass weighting is done or not. If
you select the option (default) and supply a valid \hyperref[tpr]{tpr}
file masses will be taken from there, otherwise the masses will be
deduced from the \textbf{atommass.dat} file in \textbf{GMXLIB}. This is
fine for proteins, but not necessarily for other molecules. A default
mass of 12.011 amu (carbon) is assigned to unknown atoms. You can check
whether this happend by turning on the \textbf{-debug} flag and
inspecting the \hyperref[log]{log} file.

With \textbf{-f2}, the `other structures' are taken from a second
trajectory, this generates a comparison matrix of one trajectory versus
the other.

Option \textbf{-bin} does a binary dump of the comparison matrix.

Option \textbf{-bm} produces a matrix of average bond angle deviations
analogously to the \textbf{-m} option. Only bonds between atoms in the
comparison group are considered.

\subsubsection{Files}

\ctable[pos = H, center, botcap]{llll}
{% notes
}
{% rows
\FL
option & filename & type & description
\ML
\textbf{-s} & topol.tpr & Input & Structure+mass(db):
\hyperref[tpr]{tpr} \hyperref[tpb]{tpb} \hyperref[tpa]{tpa}
\hyperref[gro]{gro} \hyperref[g96]{g96} \hyperref[pdb]{pdb}
\\\noalign{\medskip}
\textbf{-f} & traj.xtc & Input & Trajectory: \hyperref[xtc]{xtc}
\hyperref[trr]{trr} \hyperref[trj]{trj} \hyperref[gro]{gro}
\hyperref[g96]{g96} \hyperref[pdb]{pdb} cpt
\\\noalign{\medskip}
\textbf{-f2} & traj.xtc & Input, Opt. & Trajectory: \hyperref[xtc]{xtc}
\hyperref[trr]{trr} \hyperref[trj]{trj} \hyperref[gro]{gro}
\hyperref[g96]{g96} \hyperref[pdb]{pdb} cpt
\\\noalign{\medskip}
\textbf{-n} & index.ndx & Input, Opt. & Index file
\\\noalign{\medskip}
\textbf{-o} & rmsd.xvg & Output & xvgr/xmgr file
\\\noalign{\medskip}
\textbf{-mir} & rmsdmir.xvg & Output, Opt. & xvgr/xmgr file
\\\noalign{\medskip}
\textbf{-a} & avgrp.xvg & Output, Opt. & xvgr/xmgr file
\\\noalign{\medskip}
\textbf{-dist} & rmsd-dist.xvg & Output, Opt. & xvgr/xmgr file
\\\noalign{\medskip}
\textbf{-m} & rmsd.xpm & Output, Opt. & X PixMap compatible matrix file
\\\noalign{\medskip}
\textbf{-bin} & rmsd.dat & Output, Opt. & Generic data file
\\\noalign{\medskip}
\textbf{-bm} & bond.xpm & Output, Opt. & X PixMap compatible matrix file
\LL
}

\url{http://www.gromacs.org}
\href{mailto:gromacs@gromacs.org}{\texttt{gromacs@gromacs.org}}

\end{document}
